% Options for packages loaded elsewhere
\PassOptionsToPackage{unicode}{hyperref}
\PassOptionsToPackage{hyphens}{url}
\PassOptionsToPackage{dvipsnames,svgnames,x11names}{xcolor}
%
\documentclass[
  letterpaper,
  DIV=11,
  numbers=noendperiod]{scrreprt}

\usepackage{amsmath,amssymb}
\usepackage{iftex}
\ifPDFTeX
  \usepackage[T1]{fontenc}
  \usepackage[utf8]{inputenc}
  \usepackage{textcomp} % provide euro and other symbols
\else % if luatex or xetex
  \usepackage{unicode-math}
  \defaultfontfeatures{Scale=MatchLowercase}
  \defaultfontfeatures[\rmfamily]{Ligatures=TeX,Scale=1}
\fi
\usepackage{lmodern}
\ifPDFTeX\else  
    % xetex/luatex font selection
\fi
% Use upquote if available, for straight quotes in verbatim environments
\IfFileExists{upquote.sty}{\usepackage{upquote}}{}
\IfFileExists{microtype.sty}{% use microtype if available
  \usepackage[]{microtype}
  \UseMicrotypeSet[protrusion]{basicmath} % disable protrusion for tt fonts
}{}
\makeatletter
\@ifundefined{KOMAClassName}{% if non-KOMA class
  \IfFileExists{parskip.sty}{%
    \usepackage{parskip}
  }{% else
    \setlength{\parindent}{0pt}
    \setlength{\parskip}{6pt plus 2pt minus 1pt}}
}{% if KOMA class
  \KOMAoptions{parskip=half}}
\makeatother
\usepackage{xcolor}
\setlength{\emergencystretch}{3em} % prevent overfull lines
\setcounter{secnumdepth}{5}
% Make \paragraph and \subparagraph free-standing
\makeatletter
\ifx\paragraph\undefined\else
  \let\oldparagraph\paragraph
  \renewcommand{\paragraph}{
    \@ifstar
      \xxxParagraphStar
      \xxxParagraphNoStar
  }
  \newcommand{\xxxParagraphStar}[1]{\oldparagraph*{#1}\mbox{}}
  \newcommand{\xxxParagraphNoStar}[1]{\oldparagraph{#1}\mbox{}}
\fi
\ifx\subparagraph\undefined\else
  \let\oldsubparagraph\subparagraph
  \renewcommand{\subparagraph}{
    \@ifstar
      \xxxSubParagraphStar
      \xxxSubParagraphNoStar
  }
  \newcommand{\xxxSubParagraphStar}[1]{\oldsubparagraph*{#1}\mbox{}}
  \newcommand{\xxxSubParagraphNoStar}[1]{\oldsubparagraph{#1}\mbox{}}
\fi
\makeatother

\usepackage{color}
\usepackage{fancyvrb}
\newcommand{\VerbBar}{|}
\newcommand{\VERB}{\Verb[commandchars=\\\{\}]}
\DefineVerbatimEnvironment{Highlighting}{Verbatim}{commandchars=\\\{\}}
% Add ',fontsize=\small' for more characters per line
\usepackage{framed}
\definecolor{shadecolor}{RGB}{241,243,245}
\newenvironment{Shaded}{\begin{snugshade}}{\end{snugshade}}
\newcommand{\AlertTok}[1]{\textcolor[rgb]{0.68,0.00,0.00}{#1}}
\newcommand{\AnnotationTok}[1]{\textcolor[rgb]{0.37,0.37,0.37}{#1}}
\newcommand{\AttributeTok}[1]{\textcolor[rgb]{0.40,0.45,0.13}{#1}}
\newcommand{\BaseNTok}[1]{\textcolor[rgb]{0.68,0.00,0.00}{#1}}
\newcommand{\BuiltInTok}[1]{\textcolor[rgb]{0.00,0.23,0.31}{#1}}
\newcommand{\CharTok}[1]{\textcolor[rgb]{0.13,0.47,0.30}{#1}}
\newcommand{\CommentTok}[1]{\textcolor[rgb]{0.37,0.37,0.37}{#1}}
\newcommand{\CommentVarTok}[1]{\textcolor[rgb]{0.37,0.37,0.37}{\textit{#1}}}
\newcommand{\ConstantTok}[1]{\textcolor[rgb]{0.56,0.35,0.01}{#1}}
\newcommand{\ControlFlowTok}[1]{\textcolor[rgb]{0.00,0.23,0.31}{\textbf{#1}}}
\newcommand{\DataTypeTok}[1]{\textcolor[rgb]{0.68,0.00,0.00}{#1}}
\newcommand{\DecValTok}[1]{\textcolor[rgb]{0.68,0.00,0.00}{#1}}
\newcommand{\DocumentationTok}[1]{\textcolor[rgb]{0.37,0.37,0.37}{\textit{#1}}}
\newcommand{\ErrorTok}[1]{\textcolor[rgb]{0.68,0.00,0.00}{#1}}
\newcommand{\ExtensionTok}[1]{\textcolor[rgb]{0.00,0.23,0.31}{#1}}
\newcommand{\FloatTok}[1]{\textcolor[rgb]{0.68,0.00,0.00}{#1}}
\newcommand{\FunctionTok}[1]{\textcolor[rgb]{0.28,0.35,0.67}{#1}}
\newcommand{\ImportTok}[1]{\textcolor[rgb]{0.00,0.46,0.62}{#1}}
\newcommand{\InformationTok}[1]{\textcolor[rgb]{0.37,0.37,0.37}{#1}}
\newcommand{\KeywordTok}[1]{\textcolor[rgb]{0.00,0.23,0.31}{\textbf{#1}}}
\newcommand{\NormalTok}[1]{\textcolor[rgb]{0.00,0.23,0.31}{#1}}
\newcommand{\OperatorTok}[1]{\textcolor[rgb]{0.37,0.37,0.37}{#1}}
\newcommand{\OtherTok}[1]{\textcolor[rgb]{0.00,0.23,0.31}{#1}}
\newcommand{\PreprocessorTok}[1]{\textcolor[rgb]{0.68,0.00,0.00}{#1}}
\newcommand{\RegionMarkerTok}[1]{\textcolor[rgb]{0.00,0.23,0.31}{#1}}
\newcommand{\SpecialCharTok}[1]{\textcolor[rgb]{0.37,0.37,0.37}{#1}}
\newcommand{\SpecialStringTok}[1]{\textcolor[rgb]{0.13,0.47,0.30}{#1}}
\newcommand{\StringTok}[1]{\textcolor[rgb]{0.13,0.47,0.30}{#1}}
\newcommand{\VariableTok}[1]{\textcolor[rgb]{0.07,0.07,0.07}{#1}}
\newcommand{\VerbatimStringTok}[1]{\textcolor[rgb]{0.13,0.47,0.30}{#1}}
\newcommand{\WarningTok}[1]{\textcolor[rgb]{0.37,0.37,0.37}{\textit{#1}}}

\providecommand{\tightlist}{%
  \setlength{\itemsep}{0pt}\setlength{\parskip}{0pt}}\usepackage{longtable,booktabs,array}
\usepackage{calc} % for calculating minipage widths
% Correct order of tables after \paragraph or \subparagraph
\usepackage{etoolbox}
\makeatletter
\patchcmd\longtable{\par}{\if@noskipsec\mbox{}\fi\par}{}{}
\makeatother
% Allow footnotes in longtable head/foot
\IfFileExists{footnotehyper.sty}{\usepackage{footnotehyper}}{\usepackage{footnote}}
\makesavenoteenv{longtable}
\usepackage{graphicx}
\makeatletter
\newsavebox\pandoc@box
\newcommand*\pandocbounded[1]{% scales image to fit in text height/width
  \sbox\pandoc@box{#1}%
  \Gscale@div\@tempa{\textheight}{\dimexpr\ht\pandoc@box+\dp\pandoc@box\relax}%
  \Gscale@div\@tempb{\linewidth}{\wd\pandoc@box}%
  \ifdim\@tempb\p@<\@tempa\p@\let\@tempa\@tempb\fi% select the smaller of both
  \ifdim\@tempa\p@<\p@\scalebox{\@tempa}{\usebox\pandoc@box}%
  \else\usebox{\pandoc@box}%
  \fi%
}
% Set default figure placement to htbp
\def\fps@figure{htbp}
\makeatother

\KOMAoption{captions}{tableheading}
\makeatletter
\@ifpackageloaded{tcolorbox}{}{\usepackage[skins,breakable]{tcolorbox}}
\@ifpackageloaded{fontawesome5}{}{\usepackage{fontawesome5}}
\definecolor{quarto-callout-color}{HTML}{909090}
\definecolor{quarto-callout-note-color}{HTML}{0758E5}
\definecolor{quarto-callout-important-color}{HTML}{CC1914}
\definecolor{quarto-callout-warning-color}{HTML}{EB9113}
\definecolor{quarto-callout-tip-color}{HTML}{00A047}
\definecolor{quarto-callout-caution-color}{HTML}{FC5300}
\definecolor{quarto-callout-color-frame}{HTML}{acacac}
\definecolor{quarto-callout-note-color-frame}{HTML}{4582ec}
\definecolor{quarto-callout-important-color-frame}{HTML}{d9534f}
\definecolor{quarto-callout-warning-color-frame}{HTML}{f0ad4e}
\definecolor{quarto-callout-tip-color-frame}{HTML}{02b875}
\definecolor{quarto-callout-caution-color-frame}{HTML}{fd7e14}
\makeatother
\makeatletter
\@ifpackageloaded{bookmark}{}{\usepackage{bookmark}}
\makeatother
\makeatletter
\@ifpackageloaded{caption}{}{\usepackage{caption}}
\AtBeginDocument{%
\ifdefined\contentsname
  \renewcommand*\contentsname{Table of contents}
\else
  \newcommand\contentsname{Table of contents}
\fi
\ifdefined\listfigurename
  \renewcommand*\listfigurename{List of Figures}
\else
  \newcommand\listfigurename{List of Figures}
\fi
\ifdefined\listtablename
  \renewcommand*\listtablename{List of Tables}
\else
  \newcommand\listtablename{List of Tables}
\fi
\ifdefined\figurename
  \renewcommand*\figurename{Figure}
\else
  \newcommand\figurename{Figure}
\fi
\ifdefined\tablename
  \renewcommand*\tablename{Table}
\else
  \newcommand\tablename{Table}
\fi
}
\@ifpackageloaded{float}{}{\usepackage{float}}
\floatstyle{ruled}
\@ifundefined{c@chapter}{\newfloat{codelisting}{h}{lop}}{\newfloat{codelisting}{h}{lop}[chapter]}
\floatname{codelisting}{Listing}
\newcommand*\listoflistings{\listof{codelisting}{List of Listings}}
\makeatother
\makeatletter
\makeatother
\makeatletter
\@ifpackageloaded{caption}{}{\usepackage{caption}}
\@ifpackageloaded{subcaption}{}{\usepackage{subcaption}}
\makeatother

\usepackage{bookmark}

\IfFileExists{xurl.sty}{\usepackage{xurl}}{} % add URL line breaks if available
\urlstyle{same} % disable monospaced font for URLs
\hypersetup{
  pdftitle={NVivo Course},
  pdfauthor={Data Analytics Service},
  colorlinks=true,
  linkcolor={blue},
  filecolor={Maroon},
  citecolor={Blue},
  urlcolor={Blue},
  pdfcreator={LaTeX via pandoc}}


\title{NVivo Course}
\author{Data Analytics Service}
\date{2025-03-25}

\begin{document}
\maketitle

\renewcommand*\contentsname{Table of contents}
{
\hypersetup{linkcolor=}
\setcounter{tocdepth}{2}
\tableofcontents
}

\bookmarksetup{startatroot}

\chapter{Welcome}\label{welcome}

\bookmarksetup{startatroot}

\chapter{Welcome to the NVivo Course}\label{welcome-to-the-nvivo-course}

This course is designed to provide you with a \textbf{comprehensive
introduction to NVivo}, with a specific focus on working with
\textbf{interview transcripts} for qualitative analysis.

Whether you're a beginner or someone revisiting NVivo after a break,
this tutorial walks you through all the essentials---from opening the
software to importing and working with your data.

\begin{center}\rule{0.5\linewidth}{0.5pt}\end{center}

\section{👥 Who Is This Tutorial For?}\label{who-is-this-tutorial-for}

\begin{itemize}
\tightlist
\item
  Students or researchers conducting \textbf{qualitative research}
\item
  NVivo users working with \textbf{interview transcripts, focus groups,
  or documents}
\item
  Anyone new to NVivo or using it in a \textbf{university research
  setting}
\item
  Those wanting a \textbf{guided, self-paced learning resource}
\end{itemize}

No prior experience with NVivo is required.

\begin{center}\rule{0.5\linewidth}{0.5pt}\end{center}

\section{🎯 What Will You Learn?}\label{what-will-you-learn}

By the end of this course, you'll know how to:

\begin{itemize}
\tightlist
\item
  Navigate the NVivo 14 interface with confidence
\item
  Create and manage NVivo projects
\item
  Create documents directly within NVivo for notes or journaling
\item
  Prepare and import transcripts for analysis
\item
  Understand best practices for organizing qualitative data
\end{itemize}

\begin{center}\rule{0.5\linewidth}{0.5pt}\end{center}

\section{📦 What's Covered}\label{whats-covered}

This course is divided into the following foundational topics:

\begin{itemize}
\tightlist
\item
  \textbf{Opening and setting up NVivo}
\item
  \textbf{Understanding the interface}
\item
  \textbf{Creating internal documents}
\item
  \textbf{Importing and preparing transcripts}
\end{itemize}

Each chapter is concise, task-focused, and easy to follow.

\begin{center}\rule{0.5\linewidth}{0.5pt}\end{center}

\section{🧠 Who Created This Course?}\label{who-created-this-course}

This tutorial was developed by the \textbf{Data Analytics Service} at
the \textbf{University of Sheffield}.

We hope it empowers researchers with the tools they need to work
confidently with qualitative data.

\begin{center}\rule{0.5\linewidth}{0.5pt}\end{center}

\section{📚 How to Use This Tutorial}\label{how-to-use-this-tutorial}

You can move through the chapters in order or jump to the sections most
relevant to your workflow. Each page stands on its own, but they work
best as a \textbf{step-by-step guide}.

To begin, select a chapter from the sidebar or the \textbf{NVivo Basics}
section in the table of contents.

\begin{center}\rule{0.5\linewidth}{0.5pt}\end{center}

Let's get started!

\part{PART 1 · NVivo Basics}

\chapter{Opening NVivo and Creating a
Project}\label{opening-nvivo-and-creating-a-project}

\section{Opening NVivo}\label{opening-nvivo}

To open NVivo on Windows, go to the \textbf{Start Menu}, navigate to the
\textbf{QSR folder}, and select \textbf{NVivo 14}. Upon launch, you'll
be greeted with a welcome screen.

\begin{tcolorbox}[enhanced jigsaw, breakable, left=2mm, colback=white, bottomrule=.15mm, colbacktitle=quarto-callout-note-color!10!white, leftrule=.75mm, title=\textcolor{quarto-callout-note-color}{\faInfo}\hspace{0.5em}{Note}, colframe=quarto-callout-note-color-frame, coltitle=black, toptitle=1mm, opacitybacktitle=0.6, toprule=.15mm, titlerule=0mm, bottomtitle=1mm, arc=.35mm, rightrule=.15mm, opacityback=0]

Some options like ``Sign In'' and ``Collaboration Cloud'' may not apply
to institutional users (e.g., University of Sheffield), and will show a
message if not configured.

\end{tcolorbox}

\subsection{Useful Areas on the Start
Screen}\label{useful-areas-on-the-start-screen}

\begin{itemize}
\item
  \textbf{Getting Started}\\
  This section includes video tutorials and guides on:

  \begin{itemize}
  \tightlist
  \item
    Preparing transcripts for import
  \item
    Formatting survey data for NVivo
  \item
    Conducting thematic analysis
  \item
    Writing memos or journals
  \end{itemize}
\item
  \textbf{Help System}\\
  NVivo uses a \textbf{question mark (?)} icon instead of the word
  ``Help''. Look for this icon throughout the software to access help
  resources.
\item
  \textbf{Learn and Connect}\\
  Includes links to NVivo's learning materials and social platforms.
\item
  \textbf{Recent Projects}\\
  A quick-access list of projects you've recently worked on.
\end{itemize}

\section{Creating a New Project}\label{creating-a-new-project}

To start working in NVivo, you'll need to create a \textbf{project
file}. This file will hold all your: - Interview transcripts - Coding -
Annotations - Queries - Reports

\subsection{Steps}\label{steps}

\begin{enumerate}
\def\labelenumi{\arabic{enumi}.}
\tightlist
\item
  Click the \textbf{New Project} button (white button on the left).
\item
  A dialog box will appear asking you to:

  \begin{itemize}
  \tightlist
  \item
    \textbf{Name} your project
  \item
    \textbf{Choose a save location}
  \item
    Optionally, add a \textbf{description}
  \end{itemize}
\end{enumerate}

Once saved, your project will open and you can begin importing data or
documents.

\begin{tcolorbox}[enhanced jigsaw, breakable, left=2mm, colback=white, bottomrule=.15mm, colbacktitle=quarto-callout-tip-color!10!white, leftrule=.75mm, title=\textcolor{quarto-callout-tip-color}{\faLightbulb}\hspace{0.5em}{Tip}, colframe=quarto-callout-tip-color-frame, coltitle=black, toptitle=1mm, opacitybacktitle=0.6, toprule=.15mm, titlerule=0mm, bottomtitle=1mm, arc=.35mm, rightrule=.15mm, opacityback=0]

Use clear, concise project names. For example:
\texttt{qualitative\_study\_2025.nvp}

\end{tcolorbox}

\chapter{The NVivo Interface}\label{the-nvivo-interface}

\section{NVivo Environment Overview}\label{nvivo-environment-overview}

Once you create or open a project in NVivo, you'll see the main
workspace --- a clean layout divided into useful panes, tabs, and
ribbons.

To give you a proper tour, it's easier to use a \textbf{sample project}
so that the screen contains actual data and features to show.

\section{The Ribbon}\label{the-ribbon}

The \textbf{Ribbon} in NVivo is the toolbar that runs along the top. It
contains all major tools and is grouped into several tabs:

\subsection{📁 File (Blue Button)}\label{file-blue-button}

\begin{itemize}
\tightlist
\item
  Opens the \textbf{Project menu}, which allows you to save, close, or
  copy the project.
\item
  Opens a separate screen (not a ribbon).
\item
  Used occasionally, mostly for file-level operations.
\end{itemize}

\subsection{🏠 Home Tab}\label{home-tab}

\begin{itemize}
\tightlist
\item
  Central hub for common actions:

  \begin{itemize}
  \tightlist
  \item
    Coding
  \item
    Running queries
  \item
    Viewing data visualizations
  \end{itemize}
\end{itemize}

\subsection{⬇️ Import Tab}\label{import-tab}

\begin{itemize}
\tightlist
\item
  Used to bring external content into NVivo:

  \begin{itemize}
  \tightlist
  \item
    Word, PDF, images, video/audio
  \item
    Survey data (e.g., from Qualtrics)
  \item
    Bibliographic data (EndNote, Mendeley)
  \end{itemize}
\end{itemize}

\subsection{✏️ Create Tab}\label{create-tab}

\begin{itemize}
\tightlist
\item
  Lets you create content \textbf{within NVivo}:

  \begin{itemize}
  \tightlist
  \item
    Documents and memos (basic word processing tools included)
  \item
    Framework matrices
  \item
    Nodes (codes)
  \end{itemize}
\end{itemize}

\begin{tcolorbox}[enhanced jigsaw, breakable, left=2mm, colback=white, bottomrule=.15mm, colbacktitle=quarto-callout-tip-color!10!white, leftrule=.75mm, title=\textcolor{quarto-callout-tip-color}{\faLightbulb}\hspace{0.5em}{Tip}, colframe=quarto-callout-tip-color-frame, coltitle=black, toptitle=1mm, opacitybacktitle=0.6, toprule=.15mm, titlerule=0mm, bottomtitle=1mm, arc=.35mm, rightrule=.15mm, opacityback=0]

You can write directly in NVivo instead of using Word or Google Docs ---
great for journaling or annotations.

\end{tcolorbox}

\subsection{🔍 Explore Tab}\label{explore-tab}

\begin{itemize}
\tightlist
\item
  Access searches and exploration tools.
\item
  Go beyond coding with:

  \begin{itemize}
  \tightlist
  \item
    Word frequency and text search
  \item
    Cluster analysis
  \item
    Social network analysis
  \end{itemize}
\end{itemize}

\begin{tcolorbox}[enhanced jigsaw, breakable, left=2mm, colback=white, bottomrule=.15mm, colbacktitle=quarto-callout-note-color!10!white, leftrule=.75mm, title=\textcolor{quarto-callout-note-color}{\faInfo}\hspace{0.5em}{Note}, colframe=quarto-callout-note-color-frame, coltitle=black, toptitle=1mm, opacitybacktitle=0.6, toprule=.15mm, titlerule=0mm, bottomtitle=1mm, arc=.35mm, rightrule=.15mm, opacityback=0]

NVivo's search and exploration tools have been core to its power since
the early **Nud*IST** days.

\end{tcolorbox}

\subsection{📤 Share Tab}\label{share-tab}

\begin{itemize}
\tightlist
\item
  Export your work to share with others:

  \begin{itemize}
  \tightlist
  \item
    Reports as Word or Excel
  \item
    Images of visualizations
  \item
    Codebooks or queries
  \end{itemize}
\end{itemize}

Note: NVivo doesn't support direct posting to social media or emailing
content.

\subsection{🧩 Modules Tab}\label{modules-tab}

\begin{itemize}
\tightlist
\item
  Integrates with extra NVivo modules:

  \begin{itemize}
  \tightlist
  \item
    \textbf{Transcription service} (automated, paid)
  \item
    \textbf{Collaboration Cloud} (not currently supported by all
    institutions)
  \item
    \textbf{Microsoft OneDrive integration}
  \end{itemize}
\end{itemize}

\begin{tcolorbox}[enhanced jigsaw, breakable, left=2mm, colback=white, bottomrule=.15mm, colbacktitle=quarto-callout-warning-color!10!white, leftrule=.75mm, title=\textcolor{quarto-callout-warning-color}{\faExclamationTriangle}\hspace{0.5em}{Warning}, colframe=quarto-callout-warning-color-frame, coltitle=black, toptitle=1mm, opacitybacktitle=0.6, toprule=.15mm, titlerule=0mm, bottomtitle=1mm, arc=.35mm, rightrule=.15mm, opacityback=0]

Always check with your institution about data protection before using
cloud-based features like Collaboration Cloud.

\end{tcolorbox}

\section{🎯 Contextual Ribbons}\label{contextual-ribbons}

Contextual ribbons appear \textbf{based on what you're working on}.

For example, if you're editing a document, you'll see the
\textbf{Document ribbon} with relevant tools. If you're not in a
document, it disappears.

This can be confusing --- just remember: \textbf{where you click
matters}.

\section{🔧 Top-Right Utility Icons}\label{top-right-utility-icons}

At the top-right of the NVivo window, you'll see small icons:

\begin{itemize}
\tightlist
\item
  👤 \textbf{Login icon} -- Often non-functional in institutional setups
\item
  💾 \textbf{Save icon} -- Save your current project (like other apps)
\end{itemize}

\begin{center}\rule{0.5\linewidth}{0.5pt}\end{center}

This completes the visual and functional tour of NVivo's interface.

\chapter{Creating Documents in NVivo}\label{creating-documents-in-nvivo}

\section{Why Create Documents in
NVivo?}\label{why-create-documents-in-nvivo}

In most cases, you'll import transcripts or notes from external files.
But there are a few excellent reasons to create documents directly
inside NVivo:

\begin{itemize}
\tightlist
\item
  📝 \textbf{Activity Logs}: Jot down what you did in each session for
  future reference or team communication.
\item
  🔧 \textbf{Fix Transcripts}: Correct transcription errors \emph{inside
  NVivo} without re-importing or re-coding.
\item
  📚 \textbf{Appendix Material}: Useful for documenting your research
  process in a structured way.
\end{itemize}

Example log entries:

\begin{verbatim}
16 Aug – Imported volunteer interviews from Melbourne  
19 Aug – First pass of Ken’s interview: initial themes  
30 Aug – Ran search for “career”, saved query but not results  
\end{verbatim}

\begin{center}\rule{0.5\linewidth}{0.5pt}\end{center}

\section{How to Create a Document}\label{how-to-create-a-document}

There are two main ways:

\subsection{🛠️ Method 1: Use the Create
Ribbon}\label{method-1-use-the-create-ribbon}

\begin{enumerate}
\def\labelenumi{\arabic{enumi}.}
\tightlist
\item
  Go to the \textbf{Create} tab in the Ribbon.
\item
  Click the \textbf{Document} button (page icon).
\item
  The ``New Document'' dialog appears.
\end{enumerate}

\subsection{🖱️ Method 2: Right-Click in List
View}\label{method-2-right-click-in-list-view}

\begin{enumerate}
\def\labelenumi{\arabic{enumi}.}
\tightlist
\item
  Go to \textbf{Files} in the Navigation View.
\item
  Right-click in the \textbf{List View} and select:

  \begin{itemize}
  \tightlist
  \item
    \textbf{New File} \textgreater{} \textbf{New Document}
  \end{itemize}
\end{enumerate}

This approach is great when the Ribbon isn't visible or responsive.

\begin{center}\rule{0.5\linewidth}{0.5pt}\end{center}

\section{The New Document Dialog}\label{the-new-document-dialog}

You'll see two fields:

\begin{itemize}
\tightlist
\item
  \textbf{Name} -- like a filename (e.g., \texttt{Activity\ Diary})
\item
  \textbf{Description} -- longer text (e.g., \emph{``Short notes on
  tasks carried out in NVivo during data analysis.''})
\end{itemize}

There's also an \textbf{Attribute Values} tab --- this is typically
\textbf{not needed} for journal-like documents.

\begin{tcolorbox}[enhanced jigsaw, breakable, left=2mm, colback=white, bottomrule=.15mm, colbacktitle=quarto-callout-tip-color!10!white, leftrule=.75mm, title=\textcolor{quarto-callout-tip-color}{\faLightbulb}\hspace{0.5em}{Tip}, colframe=quarto-callout-tip-color-frame, coltitle=black, toptitle=1mm, opacitybacktitle=0.6, toprule=.15mm, titlerule=0mm, bottomtitle=1mm, arc=.35mm, rightrule=.15mm, opacityback=0]

Keep a single document as an internal log to track your project's
progress over time.

\end{tcolorbox}

\begin{center}\rule{0.5\linewidth}{0.5pt}\end{center}

\section{Contextual Behavior}\label{contextual-behavior}

Once you create or open a document:

\begin{itemize}
\tightlist
\item
  The \textbf{Ribbon} updates with new tabs:

  \begin{itemize}
  \tightlist
  \item
    \texttt{Document} tab
  \item
    \texttt{Edit} tab
  \end{itemize}
\item
  If nothing shows in the \textbf{List View}, click \textbf{Refresh} or
  check the filters.
\end{itemize}

\begin{tcolorbox}[enhanced jigsaw, breakable, left=2mm, colback=white, bottomrule=.15mm, colbacktitle=quarto-callout-warning-color!10!white, leftrule=.75mm, title=\textcolor{quarto-callout-warning-color}{\faExclamationTriangle}\hspace{0.5em}{Warning}, colframe=quarto-callout-warning-color-frame, coltitle=black, toptitle=1mm, opacitybacktitle=0.6, toprule=.15mm, titlerule=0mm, bottomtitle=1mm, arc=.35mm, rightrule=.15mm, opacityback=0]

Right-click menus and ribbon options depend on \emph{where you last
clicked}. If things disappear, check your focus!

\end{tcolorbox}

\chapter{Importing Interview
Transcripts}\label{importing-interview-transcripts}

\section{Why Import?}\label{why-import}

Most researchers transcribe interviews using Word or a transcription
service. Rather than copy/paste or retype content into NVivo, you can
\textbf{import multiple transcripts at once} --- clean, efficient, and
avoids rework.

\begin{center}\rule{0.5\linewidth}{0.5pt}\end{center}

\section{Recommended Workflow}\label{recommended-workflow}

We recommend a \textbf{3-step process}:

\subsection{🧱 1. Prepare NVivo to Receive
Transcripts}\label{prepare-nvivo-to-receive-transcripts}

\subsubsection{Create a Subfolder}\label{create-a-subfolder}

\begin{itemize}
\tightlist
\item
  Go to the \textbf{Files} section under \textbf{Data} in the Navigation
  View.
\item
  Create a new folder (e.g., \textbf{Transcripts}).
\item
  Give it a clear name and description.
\end{itemize}

Example: \textgreater{} Folder: \texttt{Transcripts}\\
\textgreater{} Description: \emph{``Interviews on volunteering
experiences''}

\subsubsection{\texorpdfstring{Set Up File Classifications
\emph{(Optional)}}{Set Up File Classifications (Optional)}}\label{set-up-file-classifications-optional}

Useful if you want to attach metadata (e.g., gender, role, location) to
each transcript file.

\begin{center}\rule{0.5\linewidth}{0.5pt}\end{center}

\subsection{📝 2. Prepare Your
Transcripts}\label{prepare-your-transcripts}

This step makes the import more structured and helps with later
analysis.

\subsubsection{Speaker Identifiers}\label{speaker-identifiers}

Start each speech turn with a label: - \texttt{Int}: Interviewer -
\texttt{PtA}, \texttt{PtB}, etc.: Participants\\
- Or use first names (with caution re: bias \& anonymity)

\begin{Shaded}
\begin{Highlighting}[]
\NormalTok{Int: Can you tell me about your volunteer role?}
\NormalTok{Anna: I started volunteering in 2021...}
\end{Highlighting}
\end{Shaded}

\subsubsection{Use Heading Styles}\label{use-heading-styles}

If your interviews follow a structured protocol, use Word's
\textbf{Heading 1 / 2} styles for questions or sections. This enables
\textbf{auto-coding} by structure.

\subsubsection{Add Timestamps (for Audio
Link)}\label{add-timestamps-for-audio-link}

If syncing to an audio file, insert timestamps like
\texttt{{[}00:02:15{]}} at the start of each speaker turn. NVivo can use
this to sync transcript and audio.

\begin{center}\rule{0.5\linewidth}{0.5pt}\end{center}

\subsection{📂 3. Importing the
Documents}\label{importing-the-documents}

\begin{enumerate}
\def\labelenumi{\arabic{enumi}.}
\tightlist
\item
  Navigate to \textbf{Files} in NVivo.
\item
  Click the \textbf{Import} tab in the Ribbon.
\item
  Choose \textbf{Documents} and select your transcripts from disk.
\end{enumerate}

\begin{quote}
🔍 NVivo won't know your last used folder --- navigate to the correct
directory manually.
\end{quote}

Imported documents will appear in your project under the selected
folder, ready to code, classify, or query.

\begin{center}\rule{0.5\linewidth}{0.5pt}\end{center}

\section{Wrapping Up}\label{wrapping-up}

Being intentional about how you import and format your transcripts gives
you cleaner data and smoother analysis later on.

\begin{tcolorbox}[enhanced jigsaw, breakable, left=2mm, colback=white, bottomrule=.15mm, colbacktitle=quarto-callout-tip-color!10!white, leftrule=.75mm, title=\textcolor{quarto-callout-tip-color}{\faLightbulb}\hspace{0.5em}{Tip}, colframe=quarto-callout-tip-color-frame, coltitle=black, toptitle=1mm, opacitybacktitle=0.6, toprule=.15mm, titlerule=0mm, bottomtitle=1mm, arc=.35mm, rightrule=.15mm, opacityback=0]

Organize early --- future-you will thank you!

\end{tcolorbox}




\end{document}
